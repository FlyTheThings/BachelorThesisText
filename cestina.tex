%% LyX 2.1.4 created this file.  For more info, see http://www.lyx.org/.
%% Do not edit unless you really know what you are doing.
\documentclass[a4paper,twoside,czech,czech,openright,cleardoubleempty,BCOR10mm,DIV10]{scrreprt}
\usepackage{tgpagella}
\usepackage{tgheros}
\usepackage{tgcursor}
\usepackage{eulervm}
\usepackage[T1]{fontenc}
\usepackage[utf8]{inputenc}
\usepackage{babel}
\usepackage{array}
\usepackage{fancybox}
\usepackage{calc}
\usepackage{amsmath}
\usepackage{amssymb}
\usepackage[unicode=true,pdfusetitle,
 bookmarks=true,bookmarksnumbered=false,bookmarksopen=false,
 breaklinks=false,pdfborder={0 0 1},backref=false,colorlinks=false]
 {hyperref}

\makeatletter

%%%%%%%%%%%%%%%%%%%%%%%%%%%%%% LyX specific LaTeX commands.
\pdfpageheight\paperheight
\pdfpagewidth\paperwidth

\providecommand{\LyX}{\texorpdfstring%
  {L\kern-.1667em\lower.25em\hbox{Y}\kern-.125emX\@}
  {LyX}}
\newcommand{\noun}[1]{\textsc{#1}}
%% Because html converters don't know tabularnewline
\providecommand{\tabularnewline}{\\}

%%%%%%%%%%%%%%%%%%%%%%%%%%%%%% User specified LaTeX commands.
%<-------------------------------společná nastavení------------------------------>

\usepackage{multicol}
\newcommand{\BibTeX}{{\sc Bib}\TeX}%BibTeX logo



%<-----------------------------volání stylů----------------------------------------->
% (znak % je označení komentáře: co je za ním, není aktivní)

%<--------matematické písmo--------------------------------------->

%\usepackage[helvet]{packages/sfmath}%matematika ala helvetica

%<------------------------------záhlaví stránek------------------------------------>
%\usepackage{packages/bc-headings}
\usepackage{packages/bc-fancyhdr}

%<------------------------------hlavičky kapitol------------------------------------>
%\usepackage{packages/bc-neueskapitel}
%\usepackage{packages/bc-fancychap}

\makeatother

\begin{document}

\chapter[Sazba českého textu]{Sazba českého textu v \protect\LyX{}u}

O tom, že český text musí být především gramaticky absolutně správně
se málokdo odváží pochybovat, ale stejnou (možná větší) pozornost
si zaslouží i ostatní ,,české`` prvky tištěného textu. Mám na mysli
kvalitu českého písma, dělení slov podle českých pravidel, případně
typicky česká pravidla sázení textu~\cite{Jirkovy-stranky,Menousek,Polach},
na která jsou čeští čtenáři zvyklí. Všechny tyto prvky totiž významným
způsobem ovlivňují čitelnost textu. 


\section{Výběr fontů pro český text\label{sub:Test-=00010Desk=0000E9ho-p=0000EDsma}}

Písmo používané pro tvorbu výsledného dokumentu volíme v menu \texttt{Dokument
$\rightarrow$ Nasta\-vení $\rightarrow$ Fonty.} Základní písmo
(tedy písmo použité pro psaní textu), nebo-li standardní rodina, je
ve výchozím nastavení patková antikva. Vedlejší písmo, které se používá
na nadpisy všech úrovní je bezpatkové (Sans Serif). Další vedlejší
písmo je strojopis, který se používá při výpisech kódu programů, ke
zvýraznění internetových adres, názvů počítačových souborů atp. Pozor:
jen některá, ale opravdu jen některá písma z bohaté nabídky \LyX{}u
jsou vhodná pro tvorbu dokumentů v češtině!\marginpar{\textbf{\Huge{}}%
\ovalbox{\textbf{\Huge{}!}}} \footnote{To samé platí pro fonty na různých OS, tedy i na Windows.}
Většina fontů má problém s diakritikou, třeba \usefont{T1}{ptm}{m}{n}{Times: ,,Příliš žluťoučký kůň pěl ďábelské ódy``}
a \usefont{T1}{phv}{m}{n}{Helvetica: ,,Příliš žluťoučký kůň pěl ďábelské ódy``}\normalfont  .
Na první pohled rušivě působí velké mezery za ť a ď, slovo pak vypadá
jako dvě slova. 

V tabulce~\ref{tab:Priklad-pisem} jsou vybraná použitelná písma
z nabídky \LyX{}u\footnote{Návod je napsán pro\texttt{ pdflatex}, který používá postscriptové
fonty z distribuce \LaTeX{}u. Pokud používáte jako konvertor \texttt{Xe\TeX{}}
nebo \texttt{Lua\TeX{}} budete vybírat z fontů nainstalovaných na
operačním systému. Pravidla pro výběr jsou ovšem stejná.}

\begin{table}
\caption{Příklad písem z nabídky \protect\LyX{}u použitelných pro sazbu českého textu.
\label{tab:Priklad-pisem}}


\noindent \centering{}%
\begin{tabular}{|>{\raggedright}p{0.26\textwidth}|>{\raggedright}p{0.68\textwidth}|}
\hline 
\multicolumn{2}{|c|}{\textbf{Antikva (Roman)}}\tabularnewline
\hline 
\usefont{T1}{qtm}{m}{n}{\TeX-Gyre-Termes} \\
\usefont{T1}{qpl}{m}{n}{\TeX-Gyre-Pagella} \\
\usefont{T1}{qcs}{m}{n}{\TeX-Gyre-Scholla} \\
\usefont{T1}{qbk}{m}{n}{\TeX-Gyre-Bonum} \\
\usefont{T1}{LinuxLibertineT-OsF}{m}{n}{Libertine}\\
\usefont{T1}{lmr}{m}{n}{Latin Modern Roman} & \usefont{T1}{qtm}{m}{n}{Příliš žluťoučký kůň pěl ďábelské ódy.} \\
\usefont{T1}{qpl}{m}{n}{Příliš žluťoučký kůň pěl ďábelské ódy.} \\
\usefont{T1}{qcs}{m}{n}{Příliš žluťoučký kůň pěl ďábelské ódy.} \\
\usefont{T1}{qbk}{m}{n}{Příliš žluťoučký kůň pěl ďábelské ódy.} \\
\usefont{T1}{LinuxLibertineT-OsF}{m}{n}{Příliš žluťoučký kůň pěl ďábelské ódy.}\\
\usefont{T1}{lmr}{m}{n}{ Příliš žluťoučký kůň pěl ďábelské ódy.}  \tabularnewline
\hline 
\multicolumn{2}{|c|}{\textbf{Bezšerifové písmo (Sans Serif)}}\tabularnewline
\hline 
\usefont{T1}{qhv}{m}{n}{\TeX-Gyre-Heros}\\
\usefont{T1}{qag}{m}{n}{\TeX-Gyre-Adventor}\\
\usefont{T1}{LinuxBiolinumT-OsF}{m}{n}{Biolinum}\\
\usefont{T1}{lmss}{m}{n}{Latin Modern Sans}\\
\usefont{T1}{iwona}{m}{n}{Iwona} & \usefont{T1}{qhv}{m}{n}{Příliš žluťoučký kůň pěl ďábelské ódy.} \\
\usefont{T1}{qag}{m}{n}{Příliš žluťoučký kůň pěl ďábelské ódy.}\\
\usefont{T1}{LinuxBiolinumT-OsF}{m}{n}{Příliš žluťoučký kůň pěl ďábelské ódy.}\\
\usefont{T1}{lmss}{m}{n}{Příliš žluťoučký kůň pěl ďábelské ódy.}\\
\usefont{T1}{iwona}{m}{n}{Příliš žluťoučký kůň pěl ďábelské ódy.} \tabularnewline
\hline 
\multicolumn{2}{|c|}{\textbf{Strojopis}}\tabularnewline
\hline 
\usefont{T1}{qcr}{m}{n}{\TeX-Gyre-Cursor}\\
\usefont{T1}{LinuxLibertineMonoT-TLF}{m}{n}{Libertine Mono}\\
\usefont{T1}{lmtt}{m}{n}{Latin Modern tt} & \usefont{T1}{qcr}{m}{n}{Příliš žluťoučký kůň pěl ďábelské ódy.}\\
\usefont{T1}{LinuxLibertineMonoT-TLF}{m}{n}{Příliš žluťoučký kůň pěl ďábelské ódy.}\\
\usefont{T1}{lmtt}{m}{n}{Příliš žluťoučký kůň pěl ďábelské ódy.}\tabularnewline
\hline 
\end{tabular}
\end{table}





\section{Test fontů použitých v tomto dokumentu}

Ve třech následujících odstavcích jsou nejpoužívanější řezy a tloušťky
třech základních rodin (fontů) vybraných v nastavení dokumentu. Všímejte
si, zda háčky a čárky jsou dobře umístěny a zda v~párech ďa, ťo není
příliš velká mezera mezi písmeny a také samozřejmě toho, zda daný
řez a tloušťka existuje -- tedy například zda to, co je označeno jako
kurzíva není ve skutečnosti stojaté písmo atp. \texttt{Strojopis}
ani \textsf{Sans Serif} by neměly příliš vyčnívat, jsou-li v jednom
řádku se základním písmem. Zejména \texttt{strojopis bývá někdy trochu
vyšší -} lze jej zmenšit pomocí ,,měřítka``, které je vpravo vedle
roletky, ve které vybíráte písmo.\medskip{}


\noindent \textbf{\large{}Základní písmo -- antikva, pro psaní textu:\nopagebreak[4]}{\large \par}

\noindent Stojaté: Příliš žluťoučký kůň pěl ďábelské ódy. 0123456789
\oldstylenums{0123456789}

\noindent \emph{Kurzíva: Příliš žluťoučký kůň pěl ďábelské ódy. 0123456789 }

\noindent \textbf{Tučné stojaté: Příliš žluťoučký kůň pěl ďábelské
ódy.}

\noindent \noun{Kapitálky: Příliš žluťoučký kůň pěl ďábelské ódy.}\smallskip{}


\noindent \textsf{\textbf{\large{}Vedlejší písmo -- bezpatkové, Sans
Serif, pro nadpisy:}}\textbf{\large{}\nopagebreak[4]}{\large \par}

\noindent Stojaté: \textsf{Příliš žluťoučký kůň pěl ďábelské ódy.
0123456789 \oldstylenums{0123456789}}

\noindent \textsf{\emph{Kurzíva:}}\emph{ }\textsf{\emph{Příliš žluťoučký
kůň pěl ďábelské ódy. 0123456789 }}

\noindent \textsf{\textbf{Tučné stojaté:}}\textbf{ }\textsf{\textbf{Příliš
žluťoučký kůň pěl ďábelské ódy.}}

\noindent \textsf{\noun{Kapitálky: Příliš žluťoučký kůň pěl ďábelské
ódy.}}\smallskip{}


\noindent \texttt{\textbf{\large{}Vedlejší písmo -{}- strojopis, pro
výpis kódu:}}\textbf{\large{}\nopagebreak[4]}{\large \par}

\noindent \texttt{Stojaté: Příliš žluťoučký kůň pěl ďábelské ódy. 0123456789 }

\noindent \texttt{\emph{Kurzíva: Příliš žluťoučký kůň pěl ďábelské
ódy. 0123456789 }}

\noindent \texttt{\textbf{Tučné stojaté:}}\texttt{ }\texttt{\textbf{Příliš
žluťoučký kůň pěl ďábelské ódy.}}


\section{Test matematiky}

V matematických výrazech a rovnicích by měla být použita kurzíva základního
písma pro označení proměnných ($x,y,z$), normální stojaté písmo pro
konstanty ($\mathrm{k}$), názvy funkcí ($\sin,\exp$) a~názvy operátorů
($\mathrm{rot,div,d}$)\footnote{Velmi pěkný článek o psaní matematických výrazů sepsala \noun{Juláková}~\cite{Julakova}.
Je-li ve vaší práci více matematiky, určitě si jej prostudujte!}. Rovnice\emph{ }napsaná kurzívou základního textu, například\emph{
W~=~}3\emph{F}d\emph{x}, by měla vypadat velmi podobně jako rovnice
\eqref{eq:work}:  napsaná v matematickém módu 
\begin{equation}
W=3F\mathrm{d}x.\label{eq:work}
\end{equation}


A ještě více matematiky:

\begin{equation}
\begin{aligned}H=\; & W_{SB}+W_{mv}+W_{D}-\iiint\limits _{V}X\,\mathrm{d}V-\frac{\hbar^{2}}{2m_{1}}\Delta_{1}-\sin(x)\frac{\hbar^{2}}{2m_{2}}\Delta_{2}-\frac{e^{2}}{4\pi\varepsilon_{0}|\mathbf{r}-\mathbf{R}_{1}|}\\
 & -\sum_{\begin{subarray}{c}
0<k<1000\\
\\
k\,\in\,\mathbb{N}
\end{subarray}}^{n}k^{-2}\cdot\frac{e^{2}}{4\pi\varepsilon_{0}|\mathbf{r}-\mathbf{R}_{2}|}+\prod_{R=1}^{\infty}\frac{1}{R}\cdot\frac{e^{2}}{4\pi\varepsilon_{0}|\mathbf{R}_{1}-\mathbf{R}_{2}|}
\end{aligned}
\end{equation}



\section{Test českého dělení slov}

Dělení otestujeme na textu rozděleném do úzkých odstavců. Doufám,
že si ze školy pamatujete pravidla pro dělení :). Pokud nejste spokojeni
s výsledkem, zkontrolujte, že je v Mik\TeX{}u zaškrtnuta čeština,
tedy ve Windowsech: \texttt{Start $\rightarrow$ Všechny programy
$\rightarrow$ Miktex2.9 $\rightarrow$ Settings} a na kartě \texttt{Languages}
zaškrtneme položku\texttt{ Czech}. 

\begin{multicols}{3} %


\subsection*{Červená karkulka}

Příběh vypráví o dívce přezdívané Červená karkulka, podle karkulky
(karkule byla druh čepce), kterou stále nosí. V některých verzích
jde o kapuci nebo kápi. Dívka jde lesem za svou babičkou, které nese
něco k jídlu. Vlk chce dívku sežrat, ale bojí se to udělat přímo v
lese (v některých verzích přihlížejí setkání dřevorubci). Dá se proto
s dívkou do řeči a ta mu naivně prozradí, kam má namířeno. Navrhne
jí tedy, aby nasbírala kytici květin, což udělá. Mezitím vlk přiběhne
k domu babičky, předstírá, že je Karkulka a vloudí se dovnitř. Babičku
sežere, oblékne se do jejích šatů a čeká na dívku. Když Karkulka přijde,
sežere ji také. Poté přichází dřevorubec, rozřízne vlkovi břicho a
babičku s Karkulkou, obě zcela v pořádku, zachrání. Poté naplní vlkovo
břicho těžkými kameny, čímž ho zabijí. Podle jiných verzí příběhu
vlk babičku nesežere, ale zavře ji do komory, v některých je zase
Karkulka zachráněna dřevorubcem ještě před snědením. Příběh vytváří
jasný kontrast mezi bezpečným světem vesnice a nebezpečným temným
lesem, což je pojetí v zásadě středověké, ačkoliv nejsou známé žádné
tak staré verze. Zřejmý je také morální důraz – jak je důležité nesejít
ze stezky.

Motiv vlka, který svou kořist spolkne, ale ta pak vyvázne z jeho břicha
nezraněna, se objevuje také v ruském příběhu Petr a vlk a v dalším
příběhu bratří Grimmů O vlku a~sedmi kůzlátkách, v podstatě jde o
téma staré nejméně jako Jonáš a velryba.

\end{multicols}%


\section{Odstranění předložek na konci řádku}

Aby \LyX{} nenechával předložky v, s atp. na konci řádku je třeba
mezi předložku a slovo vložit nezlomitelnou mezeru. To je možné dělat
ručně (\texttt{Ctrl + Mezerník}) nebo automaticky pomocí slavného
programu \noun{dr. Olšáka}\texttt{ vlna32.exe}\footnote{Nezlomitelná mezera se v \LaTeX{}u označuje symbolem \textasciitilde{},
proto \emph{vlna}.}. Chceme-li používat program \texttt{vlna32.exe} zkopírujeme soubory
\texttt{vlna32.exe} a \texttt{vlnapdf.bat} do složky \texttt{C:\textbackslash{}Program
Files\-\textbackslash{}Lyx\-2.1\textbackslash{}bin, v} \LyX{}u zvolíme
\texttt{Nástroje $\rightarrow$ Nastavení $\rightarrow$ Konvertory}.
V seznamu vybereme \texttt{,,\LaTeX{} (pdflatex) -> PDF (pdflatex)``}
a do \texttt{Konvertor} místo \texttt{,,pdflatex \$\$i``} napíšeme
\texttt{,,vlnapdf.bat \$\$i``} a stiskneme \texttt{}%
\ovalbox{\texttt{Změnit}} a \texttt{}%
\ovalbox{\texttt{Uložit}}. Bohužel tento postup funguje jen tehdy, když veškerý text je v jednom
dokumentu. Při použití potomků (což je i případ této šablony) je ,,ovlnkován``
jen základní dokument.
\end{document}
